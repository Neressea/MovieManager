\section{La phase de développement}

\subsection{Répartition des tâches}

La conception a constitué la première étape du projet. Nous n'avons commencé à coder qu'une fois le schéma des structures utilisées complètement finalisé. Ensuite, nous nous sommes répartis les tâches afin de réaliser parallèlement le code et les tests associés. 
La rédaction quant à elle a été réalisée en dehors des étapes de codage. 

\begin{itemize}
\item[•]{Conception}

	\begin{itemize}
		\item{Nicolas : 3h}
		\item{Vincent : 3h}
	\end{itemize}

\item[•]{Codage}

	\begin{itemize}
		\item{Nicolas : 12h}
		\item{Vincent : 15h}
	\end{itemize}

\item[•]{Tests}

	\begin{itemize}
		\item{Nicolas : 4h}
		\item{Vincent : 2h}
	\end{itemize}

\item[•]{Rédaction du rapport}
	\begin{itemize}
		\item{Nicolas : 1h}
		\item{Vincent : 2h}
	\end{itemize}
\end{itemize}

\subsection{Les problèmes rencontrés}

La partie graphique n'a pas posé de problèmes particuliers étant donné l'utilisation de la bibliothèque swing. Nous avons respecté le pattern MVC en faisant hériter les ListMovie de la classe AbstractTableModel. \\
Cependant, le modèle n'a pas été aussi facile. En effet, nous avons voulu implémenter l'algorithme H-Means vu en cours de mathématiques numériques pour la classification des données. Nous avons eu en effet beaucoup de mal car il repose sur un aléas et les résultats retournés ne sont pas toujours identiques, ce qui a rendu les tests complexes à réaliser. \\
Nous avons aussi rencontrés des problèmes dont nous n'avons pas réussi à cibler l'origine ; nous avons pu alors appréhender une nouvelle fonctionnalité de git, à savoir git reset qui nous a permis de revenir à un commit précédent. \\
Alors que la méthode fromJson permettant de charger une liste de films à partir d'un fichier formaté existe, elle a été abandonnée au cours du projet par manque de temps, afin de privilégier l'écriture des tests.

\subsection{Le lancement du logiciel}
Initialement, nous avions créé un script de compilation et d'exécution. Cependant, lorsque nous avons utilisé Eclipse, nous ne l'avons plus mis à jour et il est désormais obsolète. Il n'est plus possible de lancer le projet que par un IDE.